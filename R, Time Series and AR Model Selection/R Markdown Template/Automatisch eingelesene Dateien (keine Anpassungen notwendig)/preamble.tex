% =============== Call additional LaTeX packages ============= %

\renewcommand{\contentsname}{Inhaltsverzeichnis}
\renewcommand{\listfigurename}{Abbildungsverzeichnis}
\renewcommand{\listtablename}{Tabellenverzeichnis}
\renewcommand\refname{Literatur}

\usepackage{fancyhdr}
\pagestyle{fancy}
\setlength{\headheight}{14.5pt}
\fancyhf{}
\fancyhead[R]{\leftmark}
\fancyhead[L]{W2453 - ZAFinÖk - Projekt - SS 2025}
\fancyfoot[C]{\thepage}  
  
\usepackage{booktabs}            % for more scientific looking tables

\usepackage[nottoc]{tocbibind}   % to state list of tables and list of figures
                                 %   in the table of contents 

\usepackage[skins]{tcolorbox}
\usepackage{colortbl}
\usepackage{pdflscape}

\usepackage{mathabx}

\babelprovide[hyphenrules=ngerman-x-latest]{ngerman}

% ============= Define additional commands =================== %

% Command for R language print -> \R
\newcommand{\R}{{\fontfamily{lmss}\selectfont R }}

% Command for Python language print -> \Python
\newcommand{\Python}{{\fontfamily{lmss}\selectfont Python }}

% Command for C++ language print -> \Cpp
\newcommand{\Cpp}{{\fontfamily{lmss}\selectfont C++ }}

% Command for LaTeX name print -> \latex
\newcommand{\latex}{{\fontfamily{qpl}\selectfont LaTeX }}

% Command for MiKTeXname print -> \miktex
\newcommand{\miktex}{{\fontfamily{qpl}\selectfont MiKTeX }}

% Command for package name print -> \pkg{package_name}
\newcommand{\pkg}[1]{{\fontfamily{pbk}\fontsize{11}{12}\selectfont #1}}

% Commands to format any text using the font families lmss, pql and pbk
\newcommand{\lmss}[1]{{\fontfamily{lmss}\selectfont #1}}
\newcommand{\qpl}[1]{{\fontfamily{qpl}\selectfont #1}}
\newcommand{\pbk}[1]{{\fontfamily{pbk}\fontsize{11}{12}\selectfont #1}}

% Apply color 'violet' to text -> \violet
\newcommand{\violet}[1]{\textcolor{violet}{#1}}

% ============ Encourage LaTeX not to move figures =============== %
% ====== by Yihui Xie, Christophe Derviex and Emily Riederer ===== %
% ======        R Markdown Cookbook (2021), p. 87            ===== %

\renewcommand{\topfraction}{.85}
\renewcommand{\bottomfraction}{.7}
\renewcommand{\textfraction}{.15}
\renewcommand{\floatpagefraction}{.66}
\setcounter{topnumber}{3}
\setcounter{bottomnumber}{3}
\setcounter{totalnumber}{4}
